\documentclass[letter, 10pt]{article}
\usepackage[latin1]{inputenc}
%%\usepackage[spanish]{babel}
\usepackage{amsfonts}
\usepackage{amsmath}
\usepackage[dvips]{graphicx}
\usepackage{url}
 \usepackage{comment} 
\usepackage[top=3cm,bottom=3cm,left=3.5cm,right=3.5cm,footskip=1.5cm,headheight=1.5cm,headsep=.5cm,textheight=3cm]{geometry}



\begin{document}
\title{Inteligencia Artificial \\ \begin{Large}Estado del Arte: Aircraft Landing Scheduling Problem (ALSP)\end{Large}}
\author{Carlos Lagos Cort\'es}
\date{\today}
\maketitle


%--------------------No borrar esta secci\'on--------------------------------%
\section*{Evaluaci\'on}

\begin{tabular}{ll}
Resumen (5\%): & \underline{\hspace{2cm}} \\
Introducci\'on (5\%):  & \underline{\hspace{2cm}} \\
Definici\'on del Problema (10\%):  & \underline{\hspace{2cm}} \\
Estado del Arte (35\%):  & \underline{\hspace{2cm}} \\
Modelo Matem\'atico (20\%): &  \underline{\hspace{2cm}}\\
Conclusiones (20\%): &  \underline{\hspace{2cm}}\\
Bibliograf\'ia (5\%): & \underline{\hspace{2cm}}\\
 &  \\
\textbf{Nota Final (100\%)}:   & \underline{\hspace{2cm}}
\end{tabular}
%---------------------------------------------------------------------------%
\vspace{2cm}


\begin{abstract}
El problema de programaci\'on de aterrizaje de aeronaves (ALP) es fundamental en la gesti\'on de operaciones a\'ereas. Este informe abarca la investigaci\'on actual en ALP, explorando aspectos est\'aticos y din\'amicos, desde la formulaci\'on MILP hasta enfoques h\'ibridos. Se destaca la adaptaci\'on de soluciones est\'aticas para aeropuertos con gran tr\'afico. Varios algoritmos y heur\'isticas son mencionados, enfatizando la diversidad de soluciones. La elecci\'on del enfoque depende de las caracter\'isticas y objetivos espec\'ificos. Se subraya la evoluci\'on continua de la investigaci\'on en ALP y la importancia de abordar desaf\'ios emergentes como la gesti\'on del tr\'afico a\'ereo complejo y la sostenibilidad ambiental. En resumen, el ALP es un campo din\'amico en busca de mejorar la eficiencia y seguridad en operaciones a\'ereas.
\end{abstract}

\section{Introducci\'on}
El Aircraft Landing Scheduling Problem (ALSP) representa un reto de optimizaci\'on de gran relevancia en la industria de la aviaci\'on, y su origen se encuentra arraigado en la imperante necesidad de mejorar tanto la eficiencia operativa como la experiencia del pasajero. Este problema tiene como objetivo fundamental la reducci\'on del tiempo de espera para los pasajeros en los aeropuertos, as\'i como la disminuci\'on de los costos asociados al mantenimiento de aviones en el aire. En esencia, la misi\'on subyacente en el ALSP es abordar esta problem\'atica de manera estrat\'egica, buscando una soluci\'on que minimice tanto el tiempo de inactividad de las aeronaves en el aeropuerto como el costo relacionado con el incumplimiento de los tiempos \'optimos de aterrizaje. Este enfoque no solo tiene un impacto directo en la puntualidad de los vuelos y la eficiencia de las operaciones a\'ereas, sino que tambi\'en conlleva beneficios econ\'omicos significativos. \\

Este desaf\'io de optimizaci\'on surge de la necesidad apremiante de perfeccionar la programaci\'on de aterrizajes de aeronaves. En el contexto de este problema, se definen diversos intervalos de tiempo durante los cuales una aeronave puede efectuar su aterrizaje, y para cada uno de estos intervalos se establece un punto \'optimo de llegada. Adem\'as, se introduce un concepto cr\'itico: el costo asociado tanto a aterrizar antes como despu\'es del momento \'optimo. Es crucial destacar que, en medio de este rompecabezas de programaci\'on, existe un intervalo de tiempo predeterminado entre los aterrizajes de diferentes vuelos. \\

El prop\'osito principal de este documento es realizar un completo an\'alisis del estado del arte en el campo del Aircraft Landing Scheduling Problem (ALSP). Nuestra meta es examinar investigaciones previas y las soluciones propuestas, as\'i como explorar las distintas variantes de este problema. Adem\'as, buscamos contribuir a la resoluci\'on del ALSP al contrastar diferentes enfoques y estrategias. \\

La estructura de este informe ha sido dise\~{n}ada para proporcionar una visi\'on completa y detallada del Aircraft Landing Scheduling Problem (ALSP). Comenzaremos por establecer una base s\'olida al ofrecer una definici\'on precisa del problema. Esto implicar\'a explorar su objetivo principal y las restricciones que influyen en su resoluci\'on, lo que permitir\'a una comprensi\'on s\'olida de su alcance y complejidad. En la secci\'on denominada "Estado del Arte", nos sumergiremos en la investigaci\'on previa relacionada con el ALSP. Este an\'alisis en profundidad nos permitir\'a identificar las tendencias, enfoques y soluciones que han surgido a lo largo del tiempo. Al comprender la evoluci\'on de este campo, estaremos mejor preparados para identificar oportunidades y desaf\'ios actuales en la gesti\'on de aterrizajes de aeronaves. Posteriormente, adoptaremos un enfoque m\'as t\'ecnico al abordar el problema desde una perspectiva matem\'atica. Definiremos de manera precisa las restricciones y la funci\'on objetivo que son fundamentales para la optimizaci\'on en el ALSP. Este an\'alisis matem\'atico proporcionar\'a una base s\'olida para la evaluaci\'on y comparaci\'on de soluciones propuestas en la literatura. Finalmente, en la secci\'on de "Conclusiones", consolidaremos los aspectos m\'as destacados y cr\'iticos del informe. Aqu\'i, resumiremos los datos clave, destacaremos las tendencias m\'as significativas y ofreceremos una visi\'on general que refleje el estado actual del ALSP. Este resumen ayudar\'a a los lectores a comprender de manera concisa y precisa los aspectos m\'as importantes del problema y las investigaciones relacionadas.\\


La motivaci\'on subyacente para abordar el Aircraft Landing Scheduling Problem (ALSP) radica en la b\'usqueda constante de la industria de la aviaci\'on por mejorar la eficiencia y la calidad de los servicios ofrecidos. El ALSP tiene un impacto directo en la puntualidad de los vuelos, la reducci\'on de costos operativos y, en \'ultima instancia, en la experiencia del pasajero. Al resolver este desaf\'io de optimizaci\'on, no solo optimizamos los horarios de aterrizaje de las aeronaves, sino que tambi\'en contribuimos a minimizar las incomodidades para los viajeros y a reducir la carga ambiental causada por vuelos innecesarios. En esencia, el ALSP se presenta como un componente esencial en la mejora global de la gesti\'on de operaciones a\'ereas, impulsando as\'i un sector de la aviaci\'on m\'as eficiente, sostenible y orientado hacia la satisfacci\'on del cliente. Esta motivaci\'on nos impulsa a investigar y resolver de manera efectiva el ALSP para avanzar en la excelencia operativa en la industria de la aviaci\'on.

\section{Definici\'on del Problema}
\begin{comment}
Explicaci\'on del problema que se va a estudiar, en qu\'e consiste, cu\'ales son sus variables , restricciones y objetivo(s) de manera general (en palabras, no una formulaci\'on matem\'atica). Debe entenderse claramente el problema y qu\'e busca resolver.
Explicar si existen problemas relacionados.
Destacar, si existen, las variantes m\'as conocidas.\\
Redactar en tercera persona, sin faltas de ortograf\'ia y referenciar correctamente sus fuentes mediante el comando  \verb+\cite{ }+. Por ejemplo, para hacer referencia al art\'iculo de algoritmos h\'ibridos para problemas de satisfacci\'on 
 de restricciones~\cite{Prosser93Hybrid}.
\end{comment}


El Aircraft Landing Scheduling Problem (ALSP), en el contexto de la gesti\'on del tr\'afico a\'ereo, se define como el desaf\'io de asignar a cada aeronave que ingresa al rango de radar de control de tr\'afico a\'ereo en un aeropuerto un tiempo de aterrizaje, tambi\'en conocido como "broadcast time", y, en caso de que se utilicen m\'ultiples pistas de aterrizaje, asignarle una pista espec\'ifica. Este tiempo de aterrizaje debe estar dentro de una ventana de tiempo espec\'ifica, la cual est\'a definida por un l\'imite temprano y un l\'imite tard\'io, que pueden variar entre aeronaves. El l\'imite temprano representa el momento m\'as temprano en que una aeronave puede aterrizar si vuela a su velocidad m\'axima, mientras que el l\'imite tard\'io representa el \'ultimo momento en que puede aterrizar si vuela a su velocidad m\'as eficiente en cuanto a consumo de combustible, y adem\'as, puede requerir realizar c\'irculos durante el tiempo m\'aximo permitido. \\

Cada aeronave, en su operaci\'on normal, dispone de una velocidad de crucero preferida, la cual es la velocidad \'optima para el consumo de combustible y la eficiencia del vuelo en condiciones est\'andar. Esta velocidad de crucero preferida es un factor cr\'itico en la gesti\'on, ya que permite a las aeronaves operar de manera m\'as eficiente desde el punto de vista econ\'omico y ambiental. Cuando una aeronave recibe su "tiempo preferido" por parte del control de tr\'afico a\'ereo, significa que se le ha asignado un horario de aterrizaje que le permite mantener su velocidad de crucero preferida durante la fase de aproximaci\'on y aterrizaje. Esto es altamente deseable para las aerol\'ineas, ya que minimiza el consumo de combustible y, por lo tanto, reduce los costos operativos y la huella de carbono de la aeronave. Sin embargo, en la pr\'actica, es com\'un que las aeronaves deban realizar ajustes en su velocidad durante la aproximaci\'on y el aterrizaje debido a las instrucciones de control de tr\'afico a\'ereo o a las condiciones del tr\'afico a\'ereo en ese momento espec\'ifico. Estos ajustes pueden implicar reducir la velocidad para mantener una separaci\'on segura entre aeronaves, acelerar para evitar retrasos, o mantener una velocidad constante durante un per\'iodo de espera en el aire. Cualquier desviaci\'on de la velocidad de crucero preferida incurre en un costo adicional para la aerol\'inea, ya que implica un mayor consumo de combustible y, en algunos casos, una p\'erdida de eficiencia operativa. Este costo adicional asociado con la diferencia entre el tiempo de aterrizaje asignado y el tiempo preferido puede variar dependiendo de diversos factores, como la distancia entre la posici\'on actual de la aeronave y el aeropuerto, las condiciones meteorol\'ogicas y las instrucciones espec\'ificas de control de tr\'afico a\'ereo. La optimizaci\'on del ALSP busca minimizar estos costos adicionales al asignar horarios de aterrizaje que se acerquen lo m\'as posible a los tiempos preferidos de las aeronaves, garantizando al mismo tiempo un flujo de tr\'afico seguro y eficiente en el aeropuerto. \\

El concepto de "tiempo de separaci\'on" en el contexto del Aircraft Landing Scheduling Problem (ALSP) desempe\~{n}a un papel crucial en la gesti\'on de tr\'afico a\'ereo y la seguridad de las operaciones a\'ereas. Este tiempo de separaci\'on es el intervalo m\'inimo requerido entre el aterrizaje de una aeronave en particular y el aterrizaje de cualquier aeronave sucesiva en la misma pista de aterrizaje. La determinaci\'on de estos tiempos de separaci\'on no es arbitraria; en cambio, se basa en consideraciones aerodin\'amicas y de seguridad altamente especializadas. Uno de los factores m\'as significativos que influyen en estos tiempos es la generaci\'on de turbulencia en el aire, conocida como "v\'ortices de estela", por parte de las aeronaves que aterrizan. Los v\'ortices de estela pueden tener un efecto notable en la estabilidad de otras aeronaves que siguen la misma trayectoria de vuelo. Por lo tanto, es esencial establecer un intervalo seguro entre aterrizajes para mitigar los riesgos asociados con esta turbulencia. Cabe destacar que estas restricciones de separaci\'on tambi\'en se aplican durante las operaciones de despegue. Cuando una aeronave despega, genera v\'ortices de estela detr\'as de ella, y otras aeronaves que est\'an pr\'oximas en el espacio a\'ereo deben mantener un tiempo de separaci\'on seguro para evitar la influencia de estas turbulencias y garantizar una operaci\'on segura. En resumen, el tiempo de separaci\'on es una consideraci\'on fundamental en la planificaci\'on y la programaci\'on de aterrizajes y despegues en un aeropuerto. Estas restricciones son esenciales para salvaguardar la seguridad de las operaciones a\'ereas y asegurar que las aeronaves puedan aterrizar y despegar de manera eficiente y sin riesgos en entornos de tr\'afico a\'ereo concurridos. La optimizaci\'on del ALSP debe tener en cuenta estas restricciones de separaci\'on para garantizar que las aeronaves operen de manera segura y fluida en el aeropuerto. \\
 
La formulaci\'on propuesta por Beasly ha sido un hito fundamental en la comprensi\'on y el abordaje del Aircraft Landing Scheduling Problem (ALSP), tanto en su versi\'on est\'atica como en su versi\'on din\'amica~\cite{Beasley2004}. Beasly ha proporcionado una s\'olida base conceptual que ha permitido a la comunidad acad\'emica y a la industria a\'erea desarrollar soluciones efectivas para enfrentar este desaf\'io de manera m\'as eficiente y precisa. Lo que hace que la formulaci\'on de Beasly sea particularmente valiosa es que establece los par\'ametros esenciales y las consideraciones clave que rodean el ALSP en sus diversas variantes. En el caso est\'atico~\cite{beasley1990scheduling}, Beasly define con precisi\'on los elementos clave del problema, como los tiempos de aterrizaje, las velocidades preferidas y los tiempos de separaci\'on, lo que ha proporcionado un marco de trabajo coherente y ampliamente aceptado en la comunidad acad\'emica y de la industria. Sin embargo, Beasly tambi\'en ha contribuido significativamente al estudio del ALSP en su versi\'on din\'amica~\cite{Beasley2004}, que considera cambios en las condiciones operativas en tiempo real. Esta variante del problema agrega un nivel adicional de complejidad al ALSP y requiere estrategias de programaci\'on en tiempo real para abordar las din\'amicas cambiantes del tr\'afico a\'ereo. En resumen, la formulaci\'on de Beasly ha desempe\~{n}ado un papel fundamental al proporcionar una base s\'olida y unificada tanto para el caso est\'atico como para el caso din\'amico del ALSP. Su influencia se ha extendido tanto en la academia como en la industria a\'erea, facilitando avances significativos en la gesti\'on de operaciones a\'ereas y asegurando una base com\'un para la investigaci\'on y el desarrollo continuos en este campo cr\'itico.

\section{Estado del Arte}
\begin{comment}
La informaci\'on que describen en este punto se basa en los estudios realizados con antelaci\'on respecto al tema. Lo m\'as importante que se ha hecho hasta ahora con relaci\'on al problema. Deber\'ia responder preguntas como las siguientes:
?`cu\'ando surge?, ?`qu\'e m\'etodos se han usado para resolverlo?, ?`cu\'ales son los mejores algoritmos que se han creado hasta
la fecha?, ?`qu\'e representaciones han tenido los mejores resultados?, ?`cu\'al es la tendencia actual para resolver el problema?, tipos de movimientos, heur\'isticas, m\'etodos completos, tendencias, etc... Puede incluir gr\'aficos comparativos o explicativos.\\
\end{comment}

En las primeras etapas del estudio en el campo de la planificaci\'on y programaci\'on de vuelos, conocido como \textit{Aircraft Scheduling and Routing}, se abordaba la gesti\'on de las operaciones de las aeronaves desde una perspectiva amplia y generalizada. Este campo de investigaci\'on se centraba en aspectos fundamentales de la operaci\'on de las aeronaves, incluyendo la programaci\'on de horarios de despegue y aterrizaje, la determinaci\'on de rutas y escalas \'optimas, y la identificaci\'on de momentos estrat\'egicos para el despegue. Estas \'areas de estudio formaban la base esencial para la gesti\'on eficiente y segura de las operaciones a\'ereas.

Sin embargo, para comprender m\'as en profundidad c\'omo se abordaban estos aspectos y c\'omo se aplicaban en la pr\'actica, es crucial mencionar que las siguientes \'areas espec\'ificas de investigaci\'on se extrajeron del documento de \textbf{Maurice Pollack}~\cite{pollack1974}:

\begin{enumerate}
    \item \textbf{"Selection of Flight Departure Times" (Selecci\'on de Horarios de Salida de Vuelos):} Esta \'area se centraba en la toma de decisiones estrat\'egicas relacionadas con los horarios de partida de vuelos individuales. La elecci\'on precisa del momento en que una aeronave deb\'ia despegar era crucial para garantizar una programaci\'on eficiente y satisfacer las demandas de los pasajeros y la capacidad de la flota.

    \item \textbf{"Minimum Fleet Procedure" (Procedimiento de Flota M\'inima):} El concepto de determinar la flota de aeronaves necesaria para cubrir una programaci\'on de vuelos espec\'ifica se convirti\'o en un enfoque cr\'itico. La eficiencia operativa y la rentabilidad depend\'ian en gran medida de la capacidad de determinar la cantidad m\'inima de aeronaves requeridas para operar con eficacia.

    \item \textbf{"Aircraft Cycle Diagrams" (Diagramas de Ciclo de Aeronaves):} La representaci\'on visual de los ciclos de operaci\'on de las aeronaves, que inclu\'ian tiempos de vuelo, tiempos en tierra y procesos de mantenimiento, se utilizaba para comprender y analizar la utilizaci\'on de la flota de aeronaves a lo largo del tiempo. Estos diagramas proporcionaban una visi\'on completa de la programaci\'on y el uso de recursos.

    \item \textbf{"Evaluation of Results" (Evaluaci\'on de Resultados):} La fase de evaluaci\'on y an\'alisis de los resultados obtenidos a partir de las programaciones y rutas propuestas era esencial para la mejora continua de los procedimientos de programaci\'on de vuelos. Este enfoque permit\'ia ajustar y optimizar las operaciones en funci\'on de datos y resultados reales.
\end{enumerate}

Estas \'areas de estudio, extra\'idas del documento de Maurice Pollack~\cite{pollack1974}, representan componentes clave en la planificaci\'on y programaci\'on de vuelos en la industria de la aviaci\'on. A lo largo del tiempo, la investigaci\'on en estos campos ha evolucionado y se ha especializado a\'un m\'as, dando lugar a enfoques m\'as espec\'ificos como el \textit{Aircraft Landing Scheduling Problem (ALSP)}, que se centra en la programaci\'on de aterrizajes para garantizar la eficiencia y seguridad en los aeropuertos. \\

El art\'iculo titulado \textit{"Scheduling Aircraft Landings - The Static Case"}~\cite{beasley1990scheduling}, representa una contribuci\'on significativa en la investigaci\'on de la programaci\'on de aterrizajes de aeronaves en aeropuertos. Este trabajo aborda el desafiante problema de coordinar y secuenciar los aterrizajes de aviones de manera eficiente, un aspecto crucial para garantizar la seguridad y la fluidez de las operaciones a\'ereas en entornos aeroportuarios. La piedra angular de este art\'iculo es la presentaci\'on de una formulaci\'on de programaci\'on lineal entera mixta (MILP) dise\~{n}ada espec\'ificamente para abordar el caso est\'atico del problema de programaci\'on de aterrizajes. Esta formulaci\'on se erige como una herramienta matem\'atica poderosa que permite a los investigadores y profesionales de la aviaci\'on abordar este reto con un alto grado de precisi\'on. Fundamentada en la premisa de que cada avi\'on tiene un tiempo de llegada y un tiempo de salida deseado, la formulaci\'on tiene como objetivo principal encontrar una asignaci\'on de tiempos de aterrizaje para cada avi\'on que cumpla rigurosamente con todas las restricciones del problema. La forma que los autores tuvieron para solucionar el problema fue la siguiente:

\begin{enumerate}
    \item \textbf{Formulaci\'on MILP:} Formularon el problema como un programa lineal entero mixto (MILP) con una funci\'on objetivo que minimiza la suma de las desviaciones ponderadas de los tiempos de aterrizaje objetivo. Esta formulaci\'on proporciona una base s\'olida para encontrar soluciones \'optimas.

    \item \textbf{Refuerzo de las Relajaciones Lineales:} Para mejorar la eficacia de la resoluci\'on del MILP, introdujeron restricciones adicionales que modelan diferentes aspectos pr\'acticos del problema. Estas restricciones incluyen la elecci\'on de la funci\'on objetivo, las restricciones de precedencia entre aterrizajes, el equilibrio de la carga de trabajo de las pistas, entre otros aspectos relevantes.

    \item \textbf{Resoluci\'on \'Optima:} Resolvieron el MILP de forma \'optima utilizando una b\'usqueda en \'arbol basada en programaci\'on lineal. Este enfoque garantiza la obtenci\'on de soluciones precisas y \'optimas para el problema.

    \item \textbf{Algoritmo Heur\'istico:} Adem\'as de la resoluci\'on \'optima, presentaron un algoritmo heur\'istico eficaz para el problema. Este algoritmo ofrece soluciones de alta calidad en tiempos m\'as cortos, lo que es beneficioso para aplicaciones en tiempo real o situaciones en las que la optimizaci\'on precisa es menos cr\'itica.
\end{enumerate}

El art\'iculo enriquece su contenido al presentar resultados computacionales detallados para ambos algoritmos en varios problemas de prueba que involucran hasta 50 aviones y cuatro pistas de aterrizaje. Estos resultados son esenciales para evaluar el rendimiento de la formulaci\'on propuesta en situaciones del mundo real. Adem\'as, demuestran la eficiencia de los algoritmos de optimizaci\'on utilizados, as\'i como la capacidad de la formulaci\'on para encontrar soluciones factibles y \'optimas en un tiempo razonable. \\

Despu\'es de la investigaci\'on del caso est\'atico, los mismos autores se embarcaron en la investigaci\'on del caso din\'amico, que se describe en el art\'iculo "Displacement Problem and Dynamically Scheduling Aircraft Landings." Este enfoque din\'amico aborda la programaci\'on de aterrizajes de aviones en tiempo real, donde las decisiones deben tomarse de manera continua y adaptarse a las cambiantes condiciones operativas en el aeropuerto. Para abordar este desaf\'io, los autores definieron un "problema de desplazamiento", que se caracteriza por la toma de decisiones interconectadas y secuenciales. Luego, aplicaron este problema de desplazamiento al contexto din\'amico de programaci\'on de aterrizajes de aviones. En t\'erminos de soluci\'on, los autores emplearon una formulaci\'on de programaci\'on lineal entera mixta (MILP) que busca minimizar las desviaciones con respecto a los tiempos de aterrizaje objetivos. Adem\'as, introdujeron restricciones pr\'acticas que reflejan aspectos del entorno operativo, como la elecci\'on de la funci\'on objetivo, restricciones de precedencia y equilibrio de la carga de trabajo de las pistas, entre otros. La resoluci\'on del MILP se logr\'o mediante una b\'usqueda en \'arbol basada en programaci\'on lineal, y tambi\'en se present\'o un algoritmo heur\'istico eficaz para abordar este problema din\'amico. Los resultados computacionales se documentaron para ambos enfoques (MILP y heur\'istica) en varios problemas de prueba que involucran hasta 500 aviones y cinco pistas. Estos resultados destacan la utilidad y aplicabilidad de las estrategias propuestas en situaciones de alta complejidad y dinamismo en la programaci\'on de aterrizajes de aeronaves en aeropuertos~\cite{Beasley2004}. \\

El art\'iculo "Hybrid Method for Aircraft Landing Scheduling Based on a Job Shop Formulation,"~\cite{Bencheikh2009} publicado en 2009, propone un m\'etodo h\'ibrido para la programaci\'on de aterrizajes de aviones basado en una formulaci\'on de taller de trabajo. El objetivo del art\'iculo es estudiar el caso de m\'ultiples pistas del problema est\'atico de aterrizaje de aeronaves (ALP), donde todos los datos se conocen de antemano. En la primera parte del trabajo, se propone una formulaci\'on del problema como un modelo de programaci\'on matem\'atica para reducir el n\'umero de restricciones y dar una formulaci\'on m\'as rigurosa. Esta formulaci\'on tiene como objetivo proporcionar una representaci\'on precisa del problema ALP y establecer una base s\'olida para su resoluci\'on. En la segunda parte, se formula el ALP como un problema de programaci\'on de taller de trabajo (JSSP) basado en una representaci\'on gr\'afica. Esta formulaci\'on busca mostrar la relaci\'on entre el ALP como un problema de programaci\'on espec\'ifico y el JSSP NP-duro como una programaci\'on m\'as general. Este enfoque ayuda a comprender c\'omo los desaf\'ios de programaci\'on de aterrizaje se relacionan con problemas m\'as amplios de programaci\'on de talleres de trabajo. Finalmente, para resolver el ALP, se propone un m\'etodo h\'ibrido que combina algoritmos gen\'eticos con algoritmos de optimizaci\'on por colonia de hormigas. Este enfoque innovador aprovecha las ventajas de ambos m\'etodos para abordar eficazmente el problema de programaci\'on de aterrizajes. El art\'iculo representa una contribuci\'on m\'as reciente a la investigaci\'on del ALP, incorporando enfoques modernos de resoluci\'on y an\'alisis de problemas de programaci\'on de aterrizajes de aviones en aeropuertos~\cite{Bencheikh2009}~\cite{Colorni1994}. \\

El problema de programaci\'on de aterrizaje de aeronaves (ALP) es un desaf\'io complejo en el campo de la optimizaci\'on y la gesti\'on de operaciones a\'ereas. Dada su naturaleza NP-duro, se han desarrollado diversos enfoques y algoritmos para abordar este problema con eficacia. A continuaci\'on, se detallan algunos de los algoritmos m\'as destacados y con mejores resultados:

\begin{enumerate}
    \item \textbf{Algoritmo Eficiente para una Secuencia de Aterrizaje Dada:} Este algoritmo proporciona una soluci\'on exacta en tiempo polin\'omico para una variante del problema que se enfoca en optimizar una secuencia de aterrizaje factible en el caso de una sola pista~\cite{awasthi2013aircraft}. El tiempo de ejecuci\'on de este algoritmo es del orden de $O(N^3)$, donde $N$ representa el n\'umero de aeronaves. Esta soluci\'on exacta es fundamental para casos en los que se necesita una precisi\'on extrema en la secuencia de aterrizaje.
    
    \item \textbf{Algoritmo Eficiente para una Secuencia de Aterrizaje Dada:} Este algoritmo se destaca por proporcionar una soluci\'on eficiente que optimiza tanto las secuencias como los tiempos de aterrizaje para un conjunto de aviones. Su enfoque se centra en la coordinaci\'on precisa de las aeronaves para minimizar las demoras y garantizar una programaci\'on fluida y eficaz~\cite{awasthi2013aircraft}.
    
    \item \textbf{Heur\'isticas Simples:} Adem\'as de los enfoques exactos, se han desarrollado heur\'isticas simples para abordar el ALP. Estas heur\'isticas ofrecen soluciones aproximadas que son r\'apidas de calcular. Aunque no garantizan la \'optima global, son \'utiles en situaciones en las que se necesita una soluci\'on r\'apida y aceptable~\cite{salehipour2018algorithm}.
    
    \item \textbf{Enfoque de Programaci\'on Lineal Entera Mixta:} Este enfoque se distingue por proponer una soluci\'on exacta que involucra programaci\'on lineal entera mixta. Utiliza t\'ecnicas de programaci\'on lineal para abordar el problema de manera precisa y rigurosa. Este enfoque es particularmente valioso cuando se requiere una soluci\'on \'optima y se dispone del tiempo y los recursos necesarios para llevar a cabo una optimizaci\'on exhaustiva.
\end{enumerate}

Estos algoritmos representan una muestra de los diversos enfoques utilizados en la investigaci\'on del ALP. La elecci\'on del algoritmo m\'as adecuado depende de las caracter\'isticas espec\'ificas del problema y de los objetivos de optimizaci\'on que se persigan. Cada uno de estos enfoques contribuye al desarrollo de soluciones efectivas para la programaci\'on de aterrizajes de aeronaves, un aspecto crucial para la eficiencia y la seguridad de las operaciones a\'ereas. El ALP sigue siendo un \'area de investigaci\'on activa y en constante evoluci\'on. A medida que los aeropuertos y las demandas de tr\'afico a\'ereo contin\'uan creciendo, la necesidad de soluciones innovadoras y eficaces en la programaci\'on de aterrizajes se vuelve a\'un m\'as apremiante. Los investigadores y profesionales de la aviaci\'on est\'an abocados a explorar nuevas t\'ecnicas y enfoques que puedan abordar los desaf\'ios emergentes, como la gesti\'on de flujos de tr\'afico a\'ereo cada vez m\'as complejos y la consideraci\'on de aspectos ambientales y de sostenibilidad. En resumen, el estado actual de la investigaci\'on en el ALP refleja un campo din\'amico y vital que sigue avanzando en la b\'usqueda de soluciones para optimizar la programaci\'on de aterrizajes de aeronaves y contribuir as\'i a la eficiencia y la seguridad en las operaciones a\'ereas.

\section{Modelo Matem\'atico}
\subsection{Par\'ametros}
\begin{align*}
N & : \text{N\'umero total de pistas} \\
p & : \text{N\'umero total de aviones} \\
E_i & : \text{Tiempo m\'as temprano de aterrizaje para el avi\'on } i, \quad i = 1,2,\ldots,p \\
T_i & : \text{Tiempo ideal de aterrizaje para el avi\'on } i, \quad i = 1,2,\ldots,p \\
L_i & : \text{Tiempo m\'as tard\'io de aterrizaje para el avi\'on } i, \quad i = 1,2,\ldots,p \\
g_i & : \text{Penalizaci\'on del avi\'on } i \text{ por aterrizar antes de } T_i, \quad i = 1,2,\ldots,p \\
h_i & : \text{Penalizaci\'on del avi\'on } i \text{ por aterrizar despu\'es de } T_i, \quad i = 1,2,\ldots,p \\
S_{ij} & : \text{Separaci\'on, en unidades de tiempo, entre el aterrizaje de los aviones } i \text{ y } j, \quad i, j = 1,2,\ldots,p \\
\end{align*}

\subsection{Variables}
\begin{align*}
x_i & : \text{Tiempo de aterrizaje del avi\'on } i, \quad i = 1,2,\ldots,p \\
\delta_{ij} & : \text{Variable binaria que indica si el avi\'on } i \text{ aterriza antes que el avi\'on } j, \quad i,j = 1,2,\ldots,p \\
\end{align*}


\subsection{Funci\'on Objetivo}
Minimizar la funci\'on objetivo que considera las penalizaciones por aterrizaje temprano y tard\'io:
\[
\min \sum_{i=1}^{p} g_i\max[0,T_i - x_i] + h_i\max[0,x_i - T_i]
\]
\[
\delta_{ij} + \delta_{ji} = 1, \quad i,j = 1,2,\ldots,p \quad i \neq j
\]

\subsection{Restricciones Duras}
\begin{align*}
\text{Restricci\'on de la ventana de tiempo:} & \quad E_i \leq x_i \leq L_i, \quad i = 1,2,\ldots,p \\
\text{Separaci\'on entre aterrizajes:} & \quad x_j - x_i = S_{ij},\quad x_i < x_j \quad i, j = 1,2,\ldots,p \\
\end{align*}

\subsection{Restricciones Blandas}
\begin{align*}
\text{Restricci\'on de aterrizaje ideal:} & \quad \text{Minimizar } \{\max[0,T_i - x_i] + \max[0,x_i - T_i]\}, \quad i = 1,2,\ldots,p \\
\end{align*}
\section{Conclusiones}
\begin{comment}
Conclusiones RELEVANTES del estudio realizado. Deber\'ia responder a las preguntas: ?`todas las t\'ecnicas resuelven el mismo problema o hay algunas diferencias?, ?`En qu\'e se parecen o difieren las t\'ecnicas en el contexto del problema?, ?`qu\'e limitaciones tienen?, ?`qu\'e t\'ecnicas o estrategias son las m\'as prometedoras?, ?`existe trabajo futuro por realizar?, ?`qu\'e ideas usted propone como lineamientos para continuar con investigaciones futuras?
\end{comment}

El estudio realizado ha revelado una serie de aspectos relevantes en la investigaci\'on del problema de programaci\'on de aterrizaje de aeronaves (ALP). Aunque se han propuesto diversas t\'ecnicas para abordar el ALP, existe una distinci\'on importante entre enfoques est\'aticos y din\'amicos. En el contexto del ALP est\'atico, se destaca la efectividad de las t\'ecnicas basadas en Programaci\'on Lineal Entera Mixta (MILP) y formulaciones de taller de trabajo, que brindan soluciones exactas y eficientes. Estos enfoques han demostrado su utilidad en situaciones donde se requiere una precisi\'on extrema en la programaci\'on de aterrizajes. Sin embargo, las limitaciones surgen en t\'erminos de escalabilidad y adaptabilidad a entornos cambiantes. Por otro lado, la investigaci\'on en ALP din\'amico, como presentada por Beasley et al. (2004), ha abierto un nuevo camino en la programaci\'on de aterrizajes en tiempo real. Estos enfoques se centran en la adaptaci\'on de soluciones est\'aticas a condiciones operativas variables, lo que resulta especialmente valioso en aeropuertos de gran envergadura. Las t\'ecnicas en este contexto se enfocan en la toma de decisiones \'agiles y efectivas en tiempo real. No obstante, se reconoce que cada t\'ecnica tiene sus propias limitaciones, ya sea en t\'erminos de escalabilidad, capacidad de adaptaci\'on o tiempo de c\'omputo. La elecci\'on de la t\'ecnica m\'as adecuada depende de las caracter\'isticas espec\'ificas del problema y de los objetivos de optimizaci\'on. En cuanto a las perspectivas futuras, se sugiere la continuaci\'on de investigaciones que aborden desaf\'ios emergentes en la gesti\'on del tr\'afico a\'ereo, como la creciente complejidad y la sostenibilidad ambiental. Adem\'as, se plantea la necesidad de explorar enfoques h\'ibridos que combinen las fortalezas de las t\'ecnicas est\'aticas y din\'amicas para obtener soluciones m\'as completas y adaptables. \\

Como lineamientos para futuras investigaciones, se proponen:

\begin{enumerate}
    \item Investigar en t\'ecnicas de programaci\'on h\'ibrida que combinen enfoques est\'aticos y din\'amicos para abordar de manera m\'as efectiva los desaf\'ios del ALP.
    
    \item Evaluar el impacto de la inteligencia artificial y el aprendizaje autom\'atico en la toma de decisiones en tiempo real para el ALP.
    
    \item Continuar investigando en m\'etodos de adaptaci\'on y reprogramaci\'on din\'amica de aterrizajes en aeropuertos de gran envergadura.
\end{enumerate}


En \'ultima instancia, el ALP sigue siendo un campo en constante evoluci\'on que desempe\~{n}ar\'a un papel fundamental en la resoluci\'on de los desaf\'ios presentes y futuros en la aviaci\'on y la gesti\'on del tr\'afico a\'ereo a nivel mundial.


\section{Bibliograf\'ia}
\begin{comment}
Indicando toda la informaci\'on necesaria de acuerdo al tipo de documento revisado. Todas las referencias deben ser citadas en el documento.
\end{comment}

\bibliographystyle{plain}
\bibliography{Referencias}

\end{document} 
